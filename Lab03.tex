% Options for packages loaded elsewhere
\PassOptionsToPackage{unicode}{hyperref}
\PassOptionsToPackage{hyphens}{url}
%
\documentclass[
]{article}
\usepackage{amsmath,amssymb}
\usepackage{lmodern}
\usepackage{iftex}
\ifPDFTeX
  \usepackage[T1]{fontenc}
  \usepackage[utf8]{inputenc}
  \usepackage{textcomp} % provide euro and other symbols
\else % if luatex or xetex
  \usepackage{unicode-math}
  \defaultfontfeatures{Scale=MatchLowercase}
  \defaultfontfeatures[\rmfamily]{Ligatures=TeX,Scale=1}
\fi
% Use upquote if available, for straight quotes in verbatim environments
\IfFileExists{upquote.sty}{\usepackage{upquote}}{}
\IfFileExists{microtype.sty}{% use microtype if available
  \usepackage[]{microtype}
  \UseMicrotypeSet[protrusion]{basicmath} % disable protrusion for tt fonts
}{}
\makeatletter
\@ifundefined{KOMAClassName}{% if non-KOMA class
  \IfFileExists{parskip.sty}{%
    \usepackage{parskip}
  }{% else
    \setlength{\parindent}{0pt}
    \setlength{\parskip}{6pt plus 2pt minus 1pt}}
}{% if KOMA class
  \KOMAoptions{parskip=half}}
\makeatother
\usepackage{xcolor}
\usepackage[margin=1in]{geometry}
\usepackage{color}
\usepackage{fancyvrb}
\newcommand{\VerbBar}{|}
\newcommand{\VERB}{\Verb[commandchars=\\\{\}]}
\DefineVerbatimEnvironment{Highlighting}{Verbatim}{commandchars=\\\{\}}
% Add ',fontsize=\small' for more characters per line
\usepackage{framed}
\definecolor{shadecolor}{RGB}{248,248,248}
\newenvironment{Shaded}{\begin{snugshade}}{\end{snugshade}}
\newcommand{\AlertTok}[1]{\textcolor[rgb]{0.94,0.16,0.16}{#1}}
\newcommand{\AnnotationTok}[1]{\textcolor[rgb]{0.56,0.35,0.01}{\textbf{\textit{#1}}}}
\newcommand{\AttributeTok}[1]{\textcolor[rgb]{0.77,0.63,0.00}{#1}}
\newcommand{\BaseNTok}[1]{\textcolor[rgb]{0.00,0.00,0.81}{#1}}
\newcommand{\BuiltInTok}[1]{#1}
\newcommand{\CharTok}[1]{\textcolor[rgb]{0.31,0.60,0.02}{#1}}
\newcommand{\CommentTok}[1]{\textcolor[rgb]{0.56,0.35,0.01}{\textit{#1}}}
\newcommand{\CommentVarTok}[1]{\textcolor[rgb]{0.56,0.35,0.01}{\textbf{\textit{#1}}}}
\newcommand{\ConstantTok}[1]{\textcolor[rgb]{0.00,0.00,0.00}{#1}}
\newcommand{\ControlFlowTok}[1]{\textcolor[rgb]{0.13,0.29,0.53}{\textbf{#1}}}
\newcommand{\DataTypeTok}[1]{\textcolor[rgb]{0.13,0.29,0.53}{#1}}
\newcommand{\DecValTok}[1]{\textcolor[rgb]{0.00,0.00,0.81}{#1}}
\newcommand{\DocumentationTok}[1]{\textcolor[rgb]{0.56,0.35,0.01}{\textbf{\textit{#1}}}}
\newcommand{\ErrorTok}[1]{\textcolor[rgb]{0.64,0.00,0.00}{\textbf{#1}}}
\newcommand{\ExtensionTok}[1]{#1}
\newcommand{\FloatTok}[1]{\textcolor[rgb]{0.00,0.00,0.81}{#1}}
\newcommand{\FunctionTok}[1]{\textcolor[rgb]{0.00,0.00,0.00}{#1}}
\newcommand{\ImportTok}[1]{#1}
\newcommand{\InformationTok}[1]{\textcolor[rgb]{0.56,0.35,0.01}{\textbf{\textit{#1}}}}
\newcommand{\KeywordTok}[1]{\textcolor[rgb]{0.13,0.29,0.53}{\textbf{#1}}}
\newcommand{\NormalTok}[1]{#1}
\newcommand{\OperatorTok}[1]{\textcolor[rgb]{0.81,0.36,0.00}{\textbf{#1}}}
\newcommand{\OtherTok}[1]{\textcolor[rgb]{0.56,0.35,0.01}{#1}}
\newcommand{\PreprocessorTok}[1]{\textcolor[rgb]{0.56,0.35,0.01}{\textit{#1}}}
\newcommand{\RegionMarkerTok}[1]{#1}
\newcommand{\SpecialCharTok}[1]{\textcolor[rgb]{0.00,0.00,0.00}{#1}}
\newcommand{\SpecialStringTok}[1]{\textcolor[rgb]{0.31,0.60,0.02}{#1}}
\newcommand{\StringTok}[1]{\textcolor[rgb]{0.31,0.60,0.02}{#1}}
\newcommand{\VariableTok}[1]{\textcolor[rgb]{0.00,0.00,0.00}{#1}}
\newcommand{\VerbatimStringTok}[1]{\textcolor[rgb]{0.31,0.60,0.02}{#1}}
\newcommand{\WarningTok}[1]{\textcolor[rgb]{0.56,0.35,0.01}{\textbf{\textit{#1}}}}
\usepackage{graphicx}
\makeatletter
\def\maxwidth{\ifdim\Gin@nat@width>\linewidth\linewidth\else\Gin@nat@width\fi}
\def\maxheight{\ifdim\Gin@nat@height>\textheight\textheight\else\Gin@nat@height\fi}
\makeatother
% Scale images if necessary, so that they will not overflow the page
% margins by default, and it is still possible to overwrite the defaults
% using explicit options in \includegraphics[width, height, ...]{}
\setkeys{Gin}{width=\maxwidth,height=\maxheight,keepaspectratio}
% Set default figure placement to htbp
\makeatletter
\def\fps@figure{htbp}
\makeatother
\setlength{\emergencystretch}{3em} % prevent overfull lines
\providecommand{\tightlist}{%
  \setlength{\itemsep}{0pt}\setlength{\parskip}{0pt}}
\setcounter{secnumdepth}{-\maxdimen} % remove section numbering
\ifLuaTeX
  \usepackage{selnolig}  % disable illegal ligatures
\fi
\IfFileExists{bookmark.sty}{\usepackage{bookmark}}{\usepackage{hyperref}}
\IfFileExists{xurl.sty}{\usepackage{xurl}}{} % add URL line breaks if available
\urlstyle{same} % disable monospaced font for URLs
\hypersetup{
  pdftitle={Lab 03},
  hidelinks,
  pdfcreator={LaTeX via pandoc}}

\title{Lab 03}
\author{}
\date{\vspace{-2.5em}}

\begin{document}
\maketitle

\hypertarget{packages}{%
\section{Packages}\label{packages}}

\begin{Shaded}
\begin{Highlighting}[]
\FunctionTok{library}\NormalTok{(tidyverse)}
\FunctionTok{library}\NormalTok{(sf)}
\FunctionTok{library}\NormalTok{(dplyr)}
\end{Highlighting}
\end{Shaded}

\hypertarget{data}{%
\section{Data}\label{data}}

\begin{Shaded}
\begin{Highlighting}[]
\NormalTok{fl\_votes }\OtherTok{\textless{}{-}} \FunctionTok{st\_read}\NormalTok{(}\StringTok{"data/fl\_votes.shp"}\NormalTok{, }\AttributeTok{quiet =} \ConstantTok{TRUE}\NormalTok{)}
\NormalTok{fl\_votes }\SpecialCharTok{\%\textgreater{}\%}
  \FunctionTok{slice}\NormalTok{(}\DecValTok{1}\SpecialCharTok{:}\DecValTok{6}\NormalTok{)}
\end{Highlighting}
\end{Shaded}

\begin{verbatim}
## Simple feature collection with 6 features and 5 fields
## Geometry type: MULTIPOLYGON
## Dimension:     XY
## Bounding box:  xmin: -85.99989 ymin: 25.95675 xmax: -80.01528 ymax: 30.58427
## Geodetic CRS:  NAD83
##     county  rep16  dem16  rep20  dem20                       geometry
## 1  Alachua  46834  75820  50972  89704 MULTIPOLYGON (((-82.37389 2...
## 2    Baker  10294   2112  11911   2037 MULTIPOLYGON (((-82.10107 3...
## 3      Bay  62194  21797  66097  25614 MULTIPOLYGON (((-85.65968 3...
## 4 Bradford   8913   2924  10334   3160 MULTIPOLYGON (((-82.274 29....
## 5  Brevard 181848 119679 207883 148549 MULTIPOLYGON (((-80.49977 2...
## 6  Broward 260951 553320 333409 618752 MULTIPOLYGON (((-80.29693 2...
\end{verbatim}

\hypertarget{exercise-1}{%
\section{Exercise 1}\label{exercise-1}}

\begin{Shaded}
\begin{Highlighting}[]
\NormalTok{fl\_votes }\OtherTok{\textless{}{-}}\NormalTok{ fl\_votes }\SpecialCharTok{\%\textgreater{}\%}
  \FunctionTok{mutate}\NormalTok{(}\AttributeTok{winner20 =} \FunctionTok{if\_else}\NormalTok{(dem20 }\SpecialCharTok{\textgreater{}}\NormalTok{ rep20, }\StringTok{"Democratic"}\NormalTok{, }
                            \FunctionTok{if\_else}\NormalTok{(dem20 }\SpecialCharTok{\textless{}}\NormalTok{ rep20, }\StringTok{"Republican"}\NormalTok{, }\StringTok{"Tie"}\NormalTok{)))}
\end{Highlighting}
\end{Shaded}

\hypertarget{exercise-2}{%
\section{Exercise 2}\label{exercise-2}}

\begin{Shaded}
\begin{Highlighting}[]
\FunctionTok{library}\NormalTok{(sf)}
\FunctionTok{library}\NormalTok{(ggplot2)}

\CommentTok{\# Assuming you have already loaded the fl\_votes data into a data frame called \textquotesingle{}fl\_votes\textquotesingle{}}

\CommentTok{\# Create a color palette for the winners}
\NormalTok{color\_palette }\OtherTok{\textless{}{-}} \FunctionTok{c}\NormalTok{(}\StringTok{"\#DE0100"}\NormalTok{, }\StringTok{"\#0015BC"}\NormalTok{, }\StringTok{"\#AAAAAA"}\NormalTok{) }\CommentTok{\# Red for Republican, Blue for Democratic, Gray for Tie}

\CommentTok{\# Plot the election results by county}
\FunctionTok{ggplot}\NormalTok{() }\SpecialCharTok{+}
  \FunctionTok{geom\_sf}\NormalTok{(}\AttributeTok{data =}\NormalTok{ fl\_votes, }\FunctionTok{aes}\NormalTok{(}\AttributeTok{fill =}\NormalTok{ winner20)) }\SpecialCharTok{+}
  \FunctionTok{scale\_fill\_manual}\NormalTok{(}\AttributeTok{values =}\NormalTok{ color\_palette) }\SpecialCharTok{+}
  \FunctionTok{labs}\NormalTok{(}\AttributeTok{title =} \StringTok{"Florida\textquotesingle{}s 2020 Presidential Election Results by County"}\NormalTok{,}
       \AttributeTok{fill =} \StringTok{"Winner"}\NormalTok{) }\SpecialCharTok{+}
  \FunctionTok{theme\_minimal}\NormalTok{()}
\end{Highlighting}
\end{Shaded}

\includegraphics{Lab03_files/figure-latex/fl-plot-1-1.pdf}

\hypertarget{exercise-3}{%
\section{Exercise \#3}\label{exercise-3}}

\begin{Shaded}
\begin{Highlighting}[]
\FunctionTok{library}\NormalTok{(dplyr)}

\CommentTok{\# Assuming you have already loaded the fl\_votes data into a data frame called \textquotesingle{}fl\_votes\textquotesingle{}}

\NormalTok{fl\_props }\OtherTok{\textless{}{-}}\NormalTok{ fl\_votes }\SpecialCharTok{\%\textgreater{}\%}
  \FunctionTok{mutate}\NormalTok{(}\AttributeTok{prop\_rep16 =}\NormalTok{ rep16 }\SpecialCharTok{/}\NormalTok{ (rep16 }\SpecialCharTok{+}\NormalTok{ dem16),}
         \AttributeTok{prop\_rep20 =}\NormalTok{ rep20 }\SpecialCharTok{/}\NormalTok{ (rep20 }\SpecialCharTok{+}\NormalTok{ dem20))}
\end{Highlighting}
\end{Shaded}

\hypertarget{exercise-4}{%
\section{Exercise 4}\label{exercise-4}}

\begin{Shaded}
\begin{Highlighting}[]
\FunctionTok{library}\NormalTok{(sf)}
\FunctionTok{library}\NormalTok{(ggplot2)}

\CommentTok{\# Assuming you have already loaded the fl\_props data into a data frame called \textquotesingle{}fl\_props\textquotesingle{}}

\CommentTok{\# Plot the election results by county}
\FunctionTok{ggplot}\NormalTok{() }\SpecialCharTok{+}
  \FunctionTok{geom\_sf}\NormalTok{(}\AttributeTok{data =}\NormalTok{ fl\_props, }\FunctionTok{aes}\NormalTok{(}\AttributeTok{fill =}\NormalTok{ prop\_rep20)) }\SpecialCharTok{+}
  \FunctionTok{scale\_fill\_gradient}\NormalTok{(}\AttributeTok{low =} \StringTok{"\#0015BC"}\NormalTok{, }\AttributeTok{high =} \StringTok{"\#DE0100"}\NormalTok{, }\AttributeTok{name =} \StringTok{"Republican}\SpecialCharTok{\textbackslash{}n}\StringTok{Proportion"}\NormalTok{) }\SpecialCharTok{+}
  \FunctionTok{labs}\NormalTok{(}\AttributeTok{title =} \StringTok{"Florida\textquotesingle{}s 2020 Presidential Election Results by County"}\NormalTok{,}
       \AttributeTok{fill =} \StringTok{"Republican Proportion"}\NormalTok{) }\SpecialCharTok{+}
  \FunctionTok{theme\_minimal}\NormalTok{()}
\end{Highlighting}
\end{Shaded}

\includegraphics{Lab03_files/figure-latex/fl-plot-2-1.pdf}

\hypertarget{exercise-5}{%
\section{Exercise 5}\label{exercise-5}}

\begin{Shaded}
\begin{Highlighting}[]
\FunctionTok{library}\NormalTok{(dplyr)}

\CommentTok{\# Assuming you have already loaded the fl\_props data into a data frame called \textquotesingle{}fl\_props\textquotesingle{}}

\NormalTok{fl\_change }\OtherTok{\textless{}{-}}\NormalTok{ fl\_props }\SpecialCharTok{\%\textgreater{}\%}
  \FunctionTok{mutate}\NormalTok{(}\AttributeTok{diff\_rep =}\NormalTok{ prop\_rep20 }\SpecialCharTok{{-}}\NormalTok{ prop\_rep16)}
\end{Highlighting}
\end{Shaded}

\hypertarget{exercise-6}{%
\section{Exercise 6}\label{exercise-6}}

\begin{Shaded}
\begin{Highlighting}[]
\FunctionTok{library}\NormalTok{(sf)}
\FunctionTok{library}\NormalTok{(ggplot2)}

\CommentTok{\# Assuming you have already loaded the fl\_change data into a data frame called \textquotesingle{}fl\_change\textquotesingle{}}

\CommentTok{\# Plot the change in Republican vote share by county}
\FunctionTok{ggplot}\NormalTok{() }\SpecialCharTok{+}
  \FunctionTok{geom\_sf}\NormalTok{(}\AttributeTok{data =}\NormalTok{ fl\_change, }\FunctionTok{aes}\NormalTok{(}\AttributeTok{fill =}\NormalTok{ diff\_rep)) }\SpecialCharTok{+}
  \FunctionTok{scale\_fill\_gradient2}\NormalTok{(}\AttributeTok{low =} \StringTok{"\#0015BC"}\NormalTok{, }\AttributeTok{mid =} \StringTok{"white"}\NormalTok{, }\AttributeTok{high =} \StringTok{"\#DE0100"}\NormalTok{, }\AttributeTok{midpoint =} \DecValTok{0}\NormalTok{,}
                      \AttributeTok{name =} \StringTok{"Change in}\SpecialCharTok{\textbackslash{}n}\StringTok{Republican Vote Share"}\NormalTok{) }\SpecialCharTok{+}
  \FunctionTok{labs}\NormalTok{(}\AttributeTok{title =} \StringTok{"Change in Republican Vote Share in Florida}\SpecialCharTok{\textbackslash{}n}\StringTok{Between 2016 and 2020"}\NormalTok{,}
       \AttributeTok{fill =} \StringTok{"Change in Republican Vote Share"}\NormalTok{) }\SpecialCharTok{+}
  \FunctionTok{theme\_minimal}\NormalTok{()}
\end{Highlighting}
\end{Shaded}

\includegraphics{Lab03_files/figure-latex/fl-plot-3-1.pdf}

\hypertarget{exercise-7}{%
\section{Exercise 7}\label{exercise-7}}

The visualizations developed provide insights into the 2016 and 2020
Presidential elections in Florida. Here's what we can interpret from the
visualizations:

Visualization 1: Plot of Florida's 2020 Presidential Election Results by
County, Colored by Winner (fl-plot-1)

This visualization shows the distribution of winners (Democratic,
Republican, or Tie) in each county for the 2020 Presidential election.
It allows us to identify the counties that were won by Democrats,
Republicans, or resulted in a tie. We can observe the spatial patterns
and the dominance of certain political parties in different regions of
Florida. Visualization 2: Plot of the 2020 U.S. Presidential Election
Results by County, Colored by Republican Proportion (fl-plot-2)

This visualization represents the proportion of the two-party vote that
went to the Republican candidate in the 2020 Presidential election. It
highlights the variations in Republican support across different
counties in Florida. We can identify areas where the Republican
candidate had a higher proportion of the vote, as well as areas where
the Democratic candidate had a stronger showing. Visualization 3: Plot
of Change in Republican Vote Share in Florida Between 2016 and 2020
(fl-plot-3)

This visualization depicts the change in the Republican vote share
between the 2016 and 2020 Presidential elections. It allows us to
visualize the shift in support towards or away from the Republican
candidate in different counties. Positive values indicate an increase in
Republican vote share, while negative values indicate a decrease. We can
identify counties where there was a significant change in political
preferences over the four-year period. Limitations of these
visualizations include:

County-level granularity: The visualizations provide insights at the
county level, but they do not capture the diversity of political
opinions within each county. Political preferences can vary within a
county, and these visualizations might oversimplify the complex nature
of voting patterns.

Aggregated data: The visualizations present aggregated data for each
county, which means individual voter preferences and demographics are
not accounted for. Important factors such as population density,
urban-rural divide, and demographic shifts within counties are not
explicitly represented.

Spatial bias: The visualizations focus on the geographic distribution of
results and changes, but they do not consider other factors that might
influence voting patterns, such as socioeconomic factors, campaign
strategies, or voter turnout.

Missing contextual information: The visualizations do not provide
additional contextual information about specific events, political
campaigns, or external factors that may have influenced the election
outcomes in Florida.

To gain a more comprehensive understanding of the elections, it is
essential to complement these visualizations with further analysis,
considering additional factors and incorporating individual-level data.

\end{document}
